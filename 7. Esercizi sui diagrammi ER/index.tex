\documentclass{article}

\begin{document}

\section{Registro elettronico}

Una scuola deve creare il proprio registro elettronico. In particolare, ha la necessità di immagazzinare informazioni
riguardo i loro studenti, le informazioni che dovrà immagazzinare sono: nome, cognome, data di nascita mentre per i professori
vogliamo nome, cognome, data di nascita e la materia che insegnano (ciascun professore potrebbe insegnare più materie). \\ \\

Come qualunque registro elettronico che si rispetti, si vuole immagazzinare anche informazioni riguardo i voti
che gli studenti ricevono e in particolare oltre al voto numerico è importante avere la data di quando quest'ultimo 
è stato inserito. \\ \\

Inoltre, si vuole memorizzare anche le classi a cui sono associati gli studenti. Ogni classe può contenere un massimo
di 25 studenti (secondo le indicazioni ministeriali).

\section{Azienda di manutenzione}

L'azienda "ManutenzioneSpeciale" ha 2 tipologie di impiegati: il tecnico e il dirigente. Di ciascun tecnico e dirigente si vogliono mantenere
le informazioni "essenziali" ovvero: nome, cognome, codice fiscale, data di nascita ed età. Il tecnico può avere una specializzazione 
(come ad esempio tecnico-lavandini, tecnico-lavastoviglie) \\ \\

I dirigenti dirigono i tecnici mentre i tecnici lavorano ad uno o più progetti. Ciascun progetto inoltre ha un nome, una data di assegnazione e 
una data di scadenza entro la quale i tecnici dovrebbero completare il proprio lavoro.

\end{document}